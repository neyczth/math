\documentclass[10.5pt,a4paper,openany]{book}
\usepackage[boldfont,slantfont]{xeCJK}
\setmainfont{Euclid}
\setsansfont{Euclid}
\setmonofont{Euclid}
\setCJKmainfont{宋体}
\setCJKsansfont{黑体}
\setCJKmonofont{仿宋}
\setCJKfamilyfont{hei}{黑体}
\setCJKfamilyfont{kai}{楷体}
\newcommand{\hei}{\CJKfamily{hei}}
\newcommand{\kai}{\CJKfamily{kai}}
\usepackage{amsmath,amssymb,amsthm}
\usepackage{geometry}
\geometry{paperwidth=12cm,paperheight=20cm,left=0.5cm,right=0.5cm,top=0.5cm,bottom=0.5cm}
\def\dh{、\!\!}
\def\tdh{\text{、\!\!}}
\def\leq{\leqslant}
\def\geq{\geqslant}
\begin{document}
    \renewcommand{\baselinestretch}{1.25}\normalsize
    \setlength{\parindent}{2em}
    \setlength{\abovedisplayskip}{1pt}
    \setlength{\belowdisplayskip}{1pt}
    \begin{center}
        {\bf \kai 每日一题20170517}
    \end{center}

    {\bf \hei 题: }设$a\tdh b\tdh n$与$\dfrac{n!}{a!b!}$都是正整数. 证明: $$a+b\leq n+1+2\log_2n.$$

    {\kai (清华大学2015年全国优秀中学生数学物理体验营)}

    {\bf \hei 证: }对于任意正整数$x$, 设$v_2(x)$表示$x!$所含因子2的个数, $s_2(x)$表示$x$在二进制表示下各位之和, 下证$$v_2(x)+s_2(x)=x.$$

    {\kai 设$x=x_k\cdot 2^k+x_{k-1}\cdot 2^{k-1}+\cdots +x_1\cdot 2+x_0$, 则
    \begin{align*}
        v_2(x)=&\,\bigg[\frac{x}{2}\bigg]\!+\!\bigg[\frac{x}{2^2}\bigg]\!+\cdots +\!\bigg[\frac{x}{2^k}\bigg]\\
        =&\ \big(x_k\cdot 2^{k-1}+x_{k-1}\cdot 2^{k-2}+\cdots +x_2\cdot 2+x_1\big)\\
        +&\ \big(x_k\cdot 2^{k-2}+x_{k-1}\cdot 2^{k-3}+\cdots +x_3\cdot 2+x_2\big)\\
        +&\,\cdots +(x_k\cdot 2+x_{k-1})+(x_k)\\
        =&\ x_k\cdot\big(2^{k-1}+2^{k-2}+\cdots +1\big)\\
        +&\ x_{k-1}\cdot\big(2^{k-2}+2^{k-3}+\cdots +1\big)\\
        +&\,\cdots +x_2\cdot(2+1)+x_1\\
        =&\ x_k\cdot\big(2^k-1\big)+x_{k-1}\cdot\big(2^{k-1}-1\big)\\
        +&\,\cdots +x_2\cdot\big(2^2-1\big)+x_1\cdot (2-1)+x_0\cdot (1-1)\\
        =&\ \big(x_k\cdot 2^k+x_{k-1}\cdot 2^{k-1}+\cdots +x_1\cdot 2+x_0\big)\\
        -&\ (x_k+x_{k-1}+\cdots +x_0)\\
        =&\ x-s_2(x).
    \end{align*}
    因此$v_2(x)+s_2(x)=x$.}

    由$\dfrac{n!}{a!b!}$是正整数, 因此$$v_2(a)+v_2(b)\leq v_2(n),$$即$a-s_2(a)+b-s_2(b)\leq n-s_2(n)$.

    由$s_2(x)$定义知对任意正整数$x$有$$1\leq s_2(x)<\log_2x+1.$$因此
    \begin{align*}
        a+b\leq&\ n-s_2(n)+s_2(a)+s_2(b)\\
        <&\ n-1+(\log_2a+1)+(\log_2b+1)\\
        =&\ n+1+(\log_2a+\log_2b)\\
        <&\ n+1+2\log_2n.
    \end{align*}
\end{document}